\documentclass[a4paper,10pt]{article}
\usepackage[utf8]{inputenc}

%opening
\title{Plotting with Analysator's MayaVi interface}
\author{Otto Hannuksela}

\begin{document}

\maketitle

\tableofcontents

\newpage

\section{Plotting the grid}

\begin{verbatim}
import pytools as pt
f = pt.vlsvfile.VlsvReader('bulk.0000872.vlsv')
grid = pt.grid.MayaviGrid(f, 'rho')
\end{verbatim}

\section{How to navigate}

In order to navigate, use the mouse scroll to zoom, mouse3 to move the image and mouse 1 to tilt the 
grid.

\section{Picker options}

Analysator has implemented many picker options. These include:

\begin{enumerate}
 \item None
 \item Velocity_space
 \item Velocity_space_nearest_cellid
 \item Velocity_space_iso_surface
 \item Velocity_space_nearest_cellid_iso_surface
 \item Pitch_angle
 \item Gyrophase_angle
 \item Cut_through (See \ref{ssec:cutthrough})
\end{enumerate}


\subsection{Velocity_space and Velocity_space_nearest_cellid}

Draws the velocity space for the cell we click on. If there exists no velocity space data in the vlsv 
file for the given cellid, then using \emph{Velocity_space_nearest_cellid} is adviced, as it picks the 
nearest cellid with velocity distribution data and draws it.

Example is shown in Figure \ref{fig:vel_space}

\begin{figure}
 
\end{figure}


\end{document}
