\documentclass[a4paper,10pt]{article}
\usepackage[utf8]{inputenc}
\usepackage{float}
\usepackage{url}

%opening
\title{Analysator installation tutorial}
\author{Otto Hannuksela}

\begin{document}

\maketitle

\tableofcontents

\newpage

\section{Getting Analysator} 

Analysator is located in github under \textbf{fmihpc}.

\url{https://github.com/fmihpc/analysator}

To clone Analysator use:

\begin{verbatim}
cd ~
git clone git@github.com:fmihpc/analysator.git analysator
\end{verbatim}


\section{Installing dependencies}

\subsection{Ubuntu 14.04}

Install dependencies:

\begin{verbatim}
sudo apt-get install -y python-matplotlib python-numpy python-scipy ipython mayavi2
\end{verbatim}

\section{Testing installation}

The installation can be tested by going to the analysator folder and typing:

\begin{verbatim}
cd ~/analysator
ipython
import pytools as pt
\end{verbatim}

The full input/output should look like this (if the installation fails, the import command should 
show an error message):
\begin{verbatim}
otto@vlasiator-Latitude-E6420:~/analysator$ cd ~/analysator
otto@vlasiator-Latitude-E6420:~/analysator$ ipython
Python 2.7.6 (default, Mar 22 2014, 22:59:56) 
Type "copyright", "credits" or "license" for more information.

IPython 1.2.1 -- An enhanced Interactive Python.
?         -> Introduction and overview of IPython's features.
%quickref -> Quick reference.
help      -> Python's own help system.
object?   -> Details about 'object', use 'object??' for extra details.

In [1]: import pytools as pt

In [2]: 

\end{verbatim}


\end{document}
